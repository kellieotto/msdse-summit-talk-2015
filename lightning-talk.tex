\documentclass{beamer}

\usepackage{beamerthemesplit}
\usepackage{graphicx}
\usepackage{color, natbib, hyperref}

%theme

\usetheme{boxes} 
%\usecolortheme{seahorse} 
\setbeamertemplate{items}[default] 
%\setbeamercovered{transparent}
\setbeamertemplate{blocks}[rounded][shadow=true] 
\setbeamertemplate{navigation symbols}{} 

\mode<presentation>

\title[SET]{Student Evaluations of Teaching (Mostly) Do Not Measure Teaching Effectiveness}
\author{Kellie Ottoboni}
\institute[]{Department of Statistics, UC Berkeley \\ Berkeley Institute for Data Science}
\date{\today}

\begin{document}

\frame{\titlepage}

\section{Introduction}
\frame
{
  \frametitle{}
 \begin{center}
 \Large{ Student evaluations of teachers (SET) are used to} \\
  \begin{itemize}
  \item Quantify teaching effectiveness
  \item Compare instructors across courses
  \item Make hiring, firing, and promotion decisions  
  \end{itemize}
  \vfill
Are SET a valid measure of teaching effectiveness?
\end{center}
}

\frame
{
  \frametitle{}
  \begin{center}
  \Huge{No!}
\vfill
\Large
  \begin{itemize}
  \item SET measure ``customer satisfaction''
  \item Ratings are biased against female instructors
  \item Biases are inconsistent across universities and disciplines; impossible to ``adjust'' SET
  \end{itemize}
  \end{center}
}

\section{The Data}
\frame
{
  \frametitle{\cite{Boring2015}}
 some stuff about Anne's data
}

\frame
{
  \frametitle{\cite{MacNell2014}}
\begin{itemize}
\item Randomized, controlled experiment of students' perceptions of gender in an online course
\item Female-identified TA was rated lower in all categories (especially ``fair'' and ``prompt'') on average
\item Average ``Overall'' rating in each section
\end{itemize}
\begin{table}[htdp]
\begin{center}
\begin{tabular}{r|c|c|}
& Female-identified & Male-identified \\
\hline
Male instructor & 3.75 & 4.00\\
Female instructor & 3.625 & 4.333\\
\end{tabular}
\end{center}
\label{default}
\end{table}%


}

\section{Analysis}
\frame
{
 \frametitle{Permutation tests}
 \begin{itemize}
 \item Model-free, distribution-free test
 \item Give exact level of test without crazy assumptions, only assume the randomization of the treatments
 \item{Python package: permute \\
 \url{https://github.com/statlab/permute}
 }
 \end{itemize}

}

\section{References}
\begin{frame}
\frametitle{References}
\bibliographystyle{plainnat}
\bibliography{SETs}
\itemize
\end{frame}
\end{document}
